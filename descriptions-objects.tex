\documentclass{article}
\usepackage[varg]{newtxmath}
\usepackage{newtxtext}
\usepackage[utf8]{inputenc}
%\usepackage{amsmath}
\usepackage{natbib}
\usepackage{graphicx}
\usepackage{astrojournals} % Necesario para nombres de revistas en luis-ref.bib
\usepackage[spanish,es-minimal, english]{babel}
\usepackage{longtable}
\usepackage{fixltx2e}
\usepackage{microtype}
\usepackage{etoolbox} 
\usepackage{hyperref}
\bibliographystyle{apj}

%\title{A catalog of stationary bowshocks in the Orion Nebula }

\newcommand\ha{\ensuremath{\mathrm{H}\alpha}}
\newcommand\thC{\ensuremath{\mathrm{\theta^1\,Ori~C}}}

\begin{document}
%\maketitle

\section{Description of the objects}
\label{sec:descrip}

\subsection{Lv knot group}
\label{sec:lv}

The LV knot group is an six proplyds set that were discovered by \citet{Laques:1979}, located very close of the Trapezium and show an isotropic distribution. There is  a binary system in this group. The emission arcs were identified after. In general, these arcs are very weak, which makes it difficult to trace the edges of the shells.

\textit{158-323 (LV5).} This was previously catalogued as a round head with tail by \citet{Odell:1996}. After, it was reported as a proplyd and binary system by \citep{Ricci:2008}. An emission arc wraps around this proplyd. 
 
\textit{161-324 (LV4).} This small and bright proplyd was previously catalogued by \citet{Odell:1996, Ricci:2008}. The proplyd is surrounded by a faint but well-defined emission arc. This is located about \(4.0''\) to the southeast of 158-323.

\textit{163-317 (LV3).} This proplyd previously catalogued by \citet{Odell:1996, Ricci:2008} is surrounded by a faint and small emission arc. 

\textit{166-316. (LV2b)} This was reported as a circularly symetric source by \citet{Odell:1996}. Later, this source was catalogued as a proplyd by \citet{Ricci:2008}. 

\textit{167-317 (LV2).} This bright proplyd was previously catalogued by \citet{Odell:1994, Ricci:2008}. It exhibits a long tail. An obvious emission arc \citep{Bally:2000a} wraps around the proplyd. \citet{Bally:2000a} describe a compact microjet emerging from this proplyd. 

\textit{168-328.} This small proplyd was previously reported by \citet{Odell:1994} and \citet{Ricci:2008}. An emission arc is associated with this proplyd. The arc is much fainter than the proplyd. This object is located about \(2.1''\) to the southwest of LV1.  

\textit{168-326 (LV1).} This is a previously reported proplyd  designated 168-326S \citep{Odell:1994}. This proplyd was classified as a binary system by \citet{Ricci:2008}. An arc emision with a complex morphology is surrounded this proplyd.

\subsection{Southeast group}
\label{sec:SE}

The southeast group is located to the inside of the Orion Nebula. The central sources of their members are proplyds and their LL arcs associated have not been previously reported in the literature. Their diffuse shells are very thin. 

\textit{169-338.} This is a previously reported proplyd \citep{Odell:1994, Ricci:2008}. The small and faint proplyd is associated with a very faint but well-defined emission arc. 

\textit{177-341 (HST 1).} This very large proplyd with a long tail was previously catologed by \citet{Odell:1994, Ricci:2008}. There is probably a jet that emerges from the proplyd \citep{Bally:2000a}. We identified a well-defined but faint  emission arc associated with this proplyd.

\textit{180-331.} This was first cataloged as a star by \citet{Odell:1996}. Later, this was reported as a proplyd and a binary system by \citet{Ricci:2008}. The proplyd is surrounded by a highly asymmetric emission arc.

\textit{189-329.} The central source was classified as star by \citet{Odell:1996}. After, \citet{Ricci:2008} reported to this source as a proplyd. This object is a very faint proplyd associated with a very diffuse shell. The northern bow wing is much more extended than the southern wing. The fact that the shell is so large and diffuse, may be an indication that it is not related to the proplyd, but the fact that a small cavity is seen  around the proplyd suggest  that some degree of physical interaction is indeed accuring.

\subsection{North group}
\label{sec:N}

The north group is located inside of the orion Nebula.

\textit{142-301.} This was previously catalogued as cusp with tail  \citep{Odell:1996}. Later, this source was classified as proplyd by \citet{Bally:2000a} and \citet{Ricci:2008}. This large proplyd has one of the longest tails (\(4''\)) of any proplyd and does not have a hemispherical head. Instead, the ionization front appears to trace the disk surface, which is inclined with respect to the tail by about $55^{\circ}$. The proplyd tail points away from \(\mathrm{\theta^1\,Ori~A}\) instead of \thC~and exhibits some bends and wiggles \citep{Bally:2000a}. This proplyd is surrounded by very faint emission arc. Rather than showing continuous curvature, the emission arc appears to comprise two straight edges, which meet at a point south-east of the proplyd. The bowshock has not been previously reported in the literature. The weak shell is thicker at southern. 

\textit{154-225.} This is a previously catalogued as a elongated with diffuse boundary \citep{Odell:1996}. Later, it was reported was as a proplyd and a binary sytem by \citet{Ricci:2008}. The central source is sorrounded by a very faint emission arc.  This  emission arc has not been previously reported in the literature. The shell is lumpy.

\textit{154-240.} This large proplyd was previously reported by \citet{Bally:2000a} and \citet{Ricci:2008}. The tail of this bright proplyd has a lenght nearly \(3''\) and the protoplanetary disk is inclined seen in silhouette \citep{Bally:2000a}. We identified a emission arc associated with this proplyd. The inner edge of the shell is well-defined but is difficult to distinguish the outer edge of the shell.  

\textit{159-221.} This was first classified as a star by \citet{Odell:1996}. This same source was reported as a dark disk seen only in silhouette by \citet{Ricci:2008}.  But a faint emission rim can be seen surrounding the disk in the \ha{} image, suggesting that it is an externally ionized proplyd. We identified a previously uncatalogued emission arc associated with the central star. The outer edge of the shell is very diffuse, which  makes it difficult to trace of outer rim. The axis of the bowshock significantly deviates from the radial direction.

\textit{163-222.} This proplyd was first catalogued by \citet{Odell:1996}. There is a compact \(0.''15\) diameter disk seen nearly in face-on embedded  in this stubby  proplyd \citep{Bally:2000a, Ricci:2008}.  This source was also reported as a binary sytem \citep{Ricci:2008}. A previously uncatalogued emission arc wraps around the proplyd. The emission arc is very faint and small, but the outer and inner edges of the shell are well-defined on the eastern side. The western side of the arc is superimposed on an unrelated a brighther larger scale emission filament, making impossible to trace the arc boundaries on this side. 

\textit{165-235.} This was previously catalogued as a star by \citet{Odell:1996}. Later, It was classified as proplyd by \citet{Ricci:2008}. This proplyd is sorrounde by a previously uncatalogued emission arc. This emission arc is very faint.

\textit{170-249.} This is a previously catalogued proplyd \citet{Odell:1996, Ricci:2008}, which is sorrounded by a very faint emission arc. \citet{Ricci:2008} reported the source as a binary system.  This bright and large proplyd exhibits a long tail and an inclined disk seen in silhouette \citep{Bally:2000a}. Several  filamentary emission features with arc shape to crosses of the object. 

\textit{173-236.} This large bright proplyd was previously catalogued by \citet{Odell:1996} and \citet{Ricci:2008}. This proplyd was also catalogued by \citet{Bally:2000a} and designated 174-236. The proplyd with an irregular and wavy tail \citep[see][]{Bally:2000a} is surrounded by a very faint  emission arc. This emission arc has not been previously reported in the literature.   

\textit{178-258.}  This large and weak proplyd was catalogued by \citet{Ricci:2008}, which is surrounded by a well-defined but faint emission arc.

\subsection{Northwest group}
\label{sec:NW}

The bowshock northwest group is located to the outskirt of the Orion Nebula. The mostly the emission arcs this group have been reported in the literature, with the exception of the  073-227 arc, which it was already previously reported.    

\textit{4578-251.} This  is a very bright T Tauri star associated with an emission arc, presents a double-shell morphology. This object has an asymmetric bow shock and the shell is more extended toward the south. The emission of the outer shell is fainter than the inner shell and is unclear whether the region marked with points is part of the outer shell. This emission arc has not been previously reported in the literature. 

\textit{049-143.} Figure shows a proplyd with a very diffuse arc of emission. Its shell is thick, the wings of the bow shock are very open, and circular shaped. The inner edge is more diffuse than the outer edge and the bow is asymmetric. The  proplyd has a short tail, probably indicating that it is highly inclined and there is extinction in the center. The emission arc has not been previously reported in the literature.  
 
\textit{051-024.} We identified a previously uncataloged proplyd with its bow shock located in front of the upper end of the North Bright Bar, their shocks are aimed at Trapezium, is one of the farthest object from the Trapezium in this group. The emission shell is thin, but a second larger, and more diffuse, emission shell is seen in front of the bow shock that we have marked. It is unclear whether this outer shell is related to the object or not. If it is true the object would be similar to the nearby 072-134. The emission arc has not been previously reported in the literature. 

\textit{072-134.} This is a large proplyd surrounding an edge-on disk seen in extintion associated with an emission arc. This object is located to the south east from 051-134. It was first cataloged by \citet{Odell:1996} and it was designated 072-135. This proplyd was also reported and decribed by \citet{Bally:2000a}. Later, \citet{Ricci:2008} listed a disk seen nearly edge-on, also named 072-135. It is shell has a complex morphology with a narrow bright arc at the inner edge, together with a much a thicker and fainter shell, which is only visible on the N side.         

\textit{073-227.} \citet{Bally:2000a} reported a wind collision front associated with 073-227. The well-defined outer edge of the arc deviates significantly from the circular fit in the south wing of the bowshock.
   
\textit{074-229.} This is located at south east of 073-227, it is a T Tauri star associated with a small and faint emission bow shock. It appears to be a smaller twin of the nearby 073-227. The central star is not obviously a proplyd, but this may be because it is too small to be resolved. The emission arc has not been previously reported in the literature. The projected separation from 072-134 is about \(8''.0\), which is not significantly smaller than the expected mean projected separation between nearest neighbors given the stellar density at this distance from the Trapezium (cite). However, given that only \(\simeq30\%\) of the star in the ONC show emission arcs, the fact that 074-229 and 073-227 both show arcs means that it is likely the form a  physical binary-system.
 
\textit{101-233.} This is a first cataloged proplyd by \citet{Odell:1996}, designated 102-233, associated with a bow shock. Later, This proplyd was also cataloged by \citet{Ricci:2008}. Its cumply shell is thin and low ionization. Furthermore, this object has a clumpy shell. Several addititional broad filamentary emission features can be seen in front of the arc, but it is unclear if these are associated with the object or are a chance superposition, that also seem to be aimed toward Trapezium. The emission arc has not been previously reported in the literature. [There are references?]   

\textit{102-157.} We identified a previously uncataloged proplyd associated with a very faint emission arc. The proplyd tail is very short, indicating that it is highly inclined. 102-157 has a open bow. The southwest wing of the bowshock is crossed by an apparently unrelated east-west oriented filament, which makes it difficult to trace the emission arc on this side. The emission arc has not been previously reported in thee literature. 

\textit{106-245.} Another previously uncataloged proplyd associated with an emission arc candidate, was identified just outside HH 202 and designated 106-245. It is the second smallest arc in this group. The emission arc has not been reported previously in the literature. 

\textit{109-246.} \citet{Bally:2000a} reported a proplyd possibly associated with yet knots, which they designated 109-247, but we have 109-246, based on the accurate position. This object is located a south east from 106-245, almost in front of the 106-245 shock's. This object is within a complex region of the nebula, just outside the principal bowshock of HH 202. A chain of faint W-facing bowshocks, which may be associated with HH 202, crosses the object but the feature that we identify as the stationary emission arc is different from these since it faces ESE. The wings of the emission arc are very open.    

\textit{124-131.} This object is a circumbinary proplyd, that  was first cataloged by \citet{Odell:1996}, but they named it 124-132 and classified it as irregular. It was also cataloged by \citet{Ricci:2008}, they mention that 124-131 is a binary system and that it has a jet. \citet{Robberto:2008} reported on the discovery of the circumbinary proplyd seen in silhouette against the bright background of the Orion Nebula, they show that 124-132 is a photoevaporated disk that harbors a binary system. This source was also reported by \citet{Smith:2005a}, they show a microjet emerging perpendicular to the major axis of the disk. Figure shows a very faint emission arc, which is not detected in \ha{} but only in green and blue wideband filters, suggesting that we are seeing dust-scattered continuum.
 
\textit{206-043.} Arc candidate. This object is a bright star associated with a very faint emission bow, the arc is quite narrow. It is the  closest object to the north Bright Bar in this group. It has not been previously reported in the literature. A filamentary dark cloud is located \(3''.0\) to west, and the apparent arc may be an ilusion caused by extension of these extinction filaments.

\subsection{Southwest group}
\label{sec:SW}

The bowshock southwest group is located to the outskirts of the Orion Nebula. The mostly of its members are the tipycal LL Ory arcs, with its well-defined emission arc.  

\textit{4582-635.} This is a previously cataloged proplyd \citep{Ricci:2008}, which is surrounded by a very faint arc of emission. The faint emission arc has not been reported previously in the literature. There are hints of additional emission knots in front of the arc, but the S/N is very low and it is unclear if  they are related with the object.

\textit{000-400.} \citet{Bally:2000a} reported a wind collision fronts associated with w000-400. The central source is a previously catalogued proplyd, designated 4596-400 \citep{Ricci:2008}. The outer edge of the wings of the emission arc can be traced to much farther distances than is typical, at least 5 times the axial radius of curvature. The northern wing becomes knotty at large distances, where  the southern wing is smoothes. The bow shock has parabolic morphology and wraps around a proplyd (Fig.).   

\textit{005-514.} This proplyd was first cataloged by \citet{Odell:1996}. Later, \citet{Bally:2000a} reported a fainter and smaller wind-wind collission front, designated w005-514 and was show in \ha{} image. The lower (southeast) wing of the arc has a complex structure, with multiple overlapping filaments.

\textit{012-407.} \citet{Bally:2000a} reported a wind collision fronts associated with 012-407. The bright central star is not obviously a proplyd. The diffuse shell is very thick.  

\textit{014-414.} \citet{Bally:2000a} cataloged another wind collision front associated with 014-407. It is one of the smaller LL arcs in this group, containing a double central star. A larger scale filament which is prominent in blue/green continuum crosses the object, but seems to be unrelated. 

\textit{022-635.} This is a previously uncataloged arc of emission that wraps around a T Tauri star. The north wing of the bowshock is more extended than south wing. There are two apparently unrelated emission filaments that cross this object. 

\textit{030-524.} \citet{Bally:2000a} reported a wind-wind collision front, associated with w030-524. The central source has a tail oriented away from the Trapezium, indicating a proplyd. The northern outer edge of the asymmetric shell is very flat and it makes a sharp corner where it joins the head of the shell.

\textit{041-637.} This is a previously reported star \citep{Rio:2012} associated with a faint emission arc. There is a west faint emission arc, which is probably unrelated to the object. The emission arc has not been previously reported in the literature.

\textit{042-628.} This object was classified as proplyd by \citet{Ricci:2008}, designated 038-627. One clearly sees a previously uncataloged emission arc that wraps around this proplyd, with well defined edges. The shell shape and size are very similar to LL 1, but it is ten times fainter and lacks any evidence for a jet.    

\textit{044-527.} A faint and small emission arc associated with 044-527 was cataloged by \citet{Bally:2000a}. The central source was subsequently classified as proplyd by \citet{Ricci:2008}. This object has a jet parallel to the proplyd axis. The bowshock is very asymmetric.

\textit{LL 1 (056-519).} The T Tauri star LL Orionis is the prototype of the LL objects. Its emission arc was discovered by \citet{Gull:1979a} and is now denoted LL 1. This is a parabolic or hyperbolic bowshock that wraps around the source star \citep{Bally:2006a}. This emission arc was also reported by \citet{Bally:2000a} and as a LL Orionis-type wind-wind collision fronts LL 1 by \citet{Bally:2001}. The emission of the bowshock wings is blueshifted with repect to the backgraond nebular emission, the emission from the head of the bowshock is at a similar velocity to the nebula and shows no detectable proper motion, consistent with it being a stationary structure \citep{Henney:2013a}. LL 1 is associated with a hypersonic jet Herbig-Haro, HH 888, that arises in the T Tauri star.

\textit{065-502.} This was classified as a  non-proplyd stellar source by \citet{Odell:1996}. This star has a small protrusion which points away from Trapezium, probably indicating a proplyd tail. This proplyd tail is very short, suggesting that it is highly inclined. The source is surrounded by a very faint emission arc, which has not been previously reported in the literature. This object is located \(\approx 22.0''\) to the northeast of LL 1.    

\textit{069-601.} This was first classified as a non-proplyd stellar source by \citet{Odell:1996}. Later, a wind collision front was reported  by \citet{Bally:2000a} associated with w069-601. \citet{Ricci:2008} catalogued the central source as a proplyd. The parabolic arc emission is well-defined, which makes it easy to trace the edges of the shell. The arc shape is very similar to LL 1, but is considerably smaller. 

\textit{117-421.} This was previously classified as a non-proplyd stellar source by \citet{Odell:1996}. Later, this same object was catalogued as proplyd by \citet{Ricci:2008}. We identified a small and very faint emission arc, that wraps around this small proplyd. 

\textit{121-434.} This is  a previously catalogued proplyd \citep{Ricci:2008}, which is surrounded by a compact emission arc. The arc emission has not been  previously reported in the literature.

\subsection{West group}
\label{sec:W}

This sparsely populated group is located in the outskirts of the Orion Nebula. 

\textit{4285-458.} This is a previously unreported star associated with an emission arc. Although the outer boundary of the shell is well-defined, it is impossible to trace the shell's inner boundary due to confusion with the PSF wings of the central star. The LL arc has not been previously reported in the literature. It is the most distant emission arc from the Trapezium in this catalog. It is also much smaller than the other arcs in the west group.  

\textit{LL 3 (4408-639).} This is a previously reported LL Ori-type object \citep{Bally:2001}. An obviuos emission arc wraps around a bright star. This object exhibits a double-shell morphology. A faint emission structure protudes from the central star to the WSW, which may represent a proplyd tail, although the object has not previously been cataloged as a proplyd, similar to several sources in the northwest group.    

\textit{LL 2 (4409-242).} This LL Ori-type object \citep{Bally:2001}, is associated with the star IX~Ori. This object has a bipolar jet HH 505, which is oriented nearby perpendicular to the bowshock axis. Apart from LL 1, this is the only LL object whose kinematics have been studied via spectroscopic mapping. Unlike LL 1, the structure and kinematics of the LL arc are very asymmetrical \citep{Henney:2013a}. Only the head and northern wing of the arc are stationary structures. The southern portion of the arc has high proper motion and seems to be driven by the blueshifted jet.  

\textit{LL 4 (4427-838).} This is a previously reported LL Ori-type object \citep{Bally:2001}. The central source was reported as a proplyd and a binary system by \citet{Bally:2006a}. At first glance, this object appears to be a single arc with a very large radius of curvature. However wide closer inspection suggests that the ``wings'' of the arc are separate structures from the more curved nose region. We propose that they are not part of the true LL arc, but are driven by the bipolar jet \citep{Bally:2006a} in a similar way to the southern wing of LL 2.     

\textit{4468-605.}  \citet{Bally:2006a} show a proplyd  surrounded by a faint arc of emission with a  microjet parallel (HH 886) to the axis of the proplyd and the emission arc. Although \citet{Bally:2006a} report the jet as one-sided, a bright \ha{} filament seen at the end of the proplyd tail many represent the counterjet.  Figure shows that the arc of 4468-605 has a flared morphology, although a larger scale filament of emission, which is probably unrelated to the object, is seen in projection against the upper half of the bowshock, which makes it difficult to discern the true shape of the bowshock on this side. 

\subsection{South group}
\label{sec:S}

The bowshock south group is located in the outskirts of the Orion Nebula and southeast of the Bright Bar, several members of this group are the farthest objects from the Trapezium. The objects are generally large in this group and many have a bright inner rim.   

\textit{066-3251.} This object was classified as a proplyd by \citet{Ricci:2008} and, located at a distance \(\sim 10'\) south of the ONC core, it is one of the farthest known proplyds from the Trapezium. We have identified a faint arc associated with this proplyd, which is seen most clearly in the F555W broad band filter but is also visible in the \ha{} filter. This object is projected onto the tip of a large-scale wishing bone shaped filament, which seems to represent a local ionization front. We believe that the ``outflow'', which \citet{Ricci:2008} identify to the south of this object is merely a misidentification of part of this filament.  

\textit{116-3101.} Figure shows a proplyd associated with a small but sharply defined emission arc.  The central star is also named V488 Ori \citep{Bally:2006a}. The wings of the bow shock are very closed, the outer edge is well defined and has a circular shape since we can fit a perfect circle with the points marked.  

\textit{119-3155.} This is a binary system  associated with an emission arc. Based on the position of the outer of curvature of the arc, we assign the fainter, more northern binary component as the corresponding stellar source. It has a faint arc to the north and another to the east (Fig), but  the east arc is probably unrelated to the object, and may instead be associated with the HH 880 bowshock which passes \(15''\) to the south. This emission arc has not been previously reported in the literature.     

\textit{136-3057.} This is a T Tauri star associated with an emission arc. It is located to the north of 119-3155 and is one of the bigger objects of this group. 136-3057 has a very diffuse shell. This has not been previously reported in the literature.

\textit{138-3024.} Figure shows a T Tauri star associated with a thin shell. The arc is seen most strongly blue and green broad band filters, but it is also seen at very low contrast in the \ha{} filter. This object has not been previously reported in the literature.

\textit{203-3039.} This object has a microjet (HH 561) which was discovered in the Fabry-Pérot study of the southern Orion Nebula by \citet{Bally:2001}. Later, this HH object was described in more detail by \citet{Bally:2006a}, the microjet emerges toward the east from the variable star MY Ori, terminating in a faint bowshock at a distance of \(16''\). A fainter counterjet and bowshock are located west of the star. In addition to these previously reported features, we identify an LL-type emission arc associated with MY Ori. The faint bow is very open.

\textit{261-3018.} This is a relatively large and diffuse LL-type arc previously unreported, seen superimposed on a very complicated region of overlapping flows. We tentatively identify the star 261-3018 as the source, although the star (264-3016) is another possible candidate. In the \ha{} the 261-3018 star (which is the same source as 262-3018 reported by \citealp{Bally:2006a})  shows a linear protrusion of length \(\simeq1''\) towards the southwest. This is unlikely to be a proplyd tail, since it is not aligned with the radial direction from the Trapezium, and instead may be a microjet. Indeed, a series of faint knots and bowshocks extend for \(30''\) in the same direction. The LL-type arc is very flat and shows a bright rim at its inner edge. It extends further to the east than to the west. The object is seen in projection superimposed on the HH 502 flow but shows no evidence of physical interaction with this flow. 

\textit{266-558.} \citet{Bally:2000a} catalogued a wind collision front associated with w266-558. Figure shows emission shell that wraps around the source. The source is a previously catalogued proplyd by \citet{Ricci:2008}. w266-558 presents a double-shell morphology, with knotty bow wings. The bow wings are very open. Although, we have assigned this object to the south group, it is located much closer to the Trapezium than the other members, in a region lacking in other arcs.

\textit{305-811.} Previously classified as LL-type object by \citet{Bally:2006a}. The emission arc is extremely faint and asymmetric. The star shows  a faint protrusion to the southeast in the \ha{} filter, probably indicating a proplyd tail, although the source has not previously been catalogued as a proplyd.    

\textit{308-3036.} \citet{Bally:2006a} reported a bright star, that is surrounded by a faint arc of emission, located to the west of LL 6. This object has a nearly circular inner shock, the circle fit is roughly centered on the star. The circular silhouette (see Fig.) may trace a wind cavity, or possibly a dusty envelope surrounding the star \citep{Bally:2006a}.       

\textit{LL 5 (315-816).} This is a LL Ori type object identified in the outskirts of the Orion Nebula, it was first reported by \citet{Bally:2001a}. Later, it was also catalogued by \citet{Bally:2006a} and they mentioned that LL 5 is associated with the  V1559 Ori star. This object has a short jet (HH 875 \citealp{Bally:2006a}) emerging from the proplyd toward the northeast.  The emission arc wraps around a proplyd, and was designated LLP 315-816 or 314-816. This object has a double-shell morphology with a bright inner rim.
       
\textit{LL 6 (329-3021).} \citep{Bally:2001} identified a LL Orionis-type wind-wind collision fronts LL 6 in the outskirts of the Orion Nebula. LL 6 is associated with the star NX Ori, which is surrounded by a large arc of emission \citep{Bally:2006a}. The wings of the bowshock are very open and extended. This object has a prominent, one-sided jet, oriented perpendicular to the axis of the LL arc. It is difficult to distinguish between the wings of the LL arc and the jet-driven bowshocks.    

\textit{344-3020.} \citet{Bally:2006a} catalogued a proplyd associated with an emission arc, designated LL 344-3020. This object is very faint and has a bipolar jet at P.A. = $45^{\circ}$.     

\textit{LL 7 (351-3349).} This object is a LL-type bowshock previously reported by \citet{Bally:2001}. The central source is a previously catalogued proplyd \citep{Ricci:2008}, designated 351-3349. This object has a jet at P.A. $\sim 80^{\circ}$ perpendicular to LL arc axis. The  bowshock  wings are very open, and it is one of the most distant emission arcs from the Trapezium in this catalog. This object is larger than we show, the bowshock from jet overlap wings of LL arc (see \citealp{Bally:2001}).   

\textit{362-3137.} This is a previously uncatalogued proplyd associated with an emission arc. Howeber, \citet{Rio:2009} catalogued the central source as a star. The source is a double star. It is located at north of LL 7. This object seems to show a double-shell morphology. There is a filament to south probably unrelated to the object. This emission arc has not been previously reported in the literature. 

%Instead of following a circle the bow shock seems to have a geometry delimited by a line perpendicular to the RA axis and a line with a positive slope in the upper part

\subsection{Interproplyd shells}
\label{sec:inter}

The interproplyd shell is a  group of small proplyds, which they are associated with small emission arcs which are formed by the interactions between two individual photoevaporation flows.

\textit{066-652.} This object was previously classified as irregular by \citet{Odell:1996}. It  was  cataloged as a proplyd and binary system by \citet{Ricci:2008}. An small emission arc is associated with this proplyd. 
    
\textit{160-350.} This was previosly reported as a cusp with tail by \citet{Odell:1996} and designated 159-350. 

\textit{162-456.}

\textit{168-326N.}

\textit{173-342.}

\textit{175-321.}

\textit{204-330.}

\subsection{Probably not shells}
\label{sec:notshell}

\textit{131-046.} 

\textit{212-400.}


\bibliography{luis-ref}

\end{document}
