% \documentclass{article}
% \usepackage[utf8]{inputenc}
% \usepackage{amsmath}
% \usepackage{natbib}
% \usepackage{graphicx}
% \usepackage{astrojournals} % Necesario para nombres de revistas en luis-ref.bib
% \usepackage[spanish, es-minimal]{babel}
% \bibliographystyle{apj}
%\title{Catalog of stationary bowshock arcs in the Orion Nebula}

% \author{
%   Alumno: Luis Angel Gutiérrez Soto\\
%   Tutor: Dr. William Henney
% }

% \begin{document}
% \maketitle

\label{chap:introduction}

Este trabajo es el resultado de la identificación y detección de arcos de emisión en los alrededores de \thC{} (O7 V) en la Nebulosa de Orión. Dichos arcos radiativos pueden ser interpretados como choques de proa que se producen por la interacción de dos vientos, esta interacción provoca que se forme una estructura con un doble choque y que la cáscara chocada está en equilibrio, es así que se puede considerar que los arcos son estacionarios. Por otro lado se tiene que en el interior de la Nebulosa de Orión; los arcos están asociados a proplyds, mientras que en regiones más externas y lejanas del grupo de las estrellas masivas llamado el Trapecio, los arcos están asociados a estrellas jóvenes de baja masa (estrellas T-Tauri). Por tanto uno de los objetivo de esta tesis, además de hacer un catálogo de choques de proa, es plantear características diferenciadoras entre los choques de los proplyds cerca de \thC{}, con los choques ubicados en las afueras de la Nebulosa de Orión, usando parámetros observacionales para estudiar sus formas y tamaños y parámetros astrofísicos para estudiar la naturaleza de la cáscara chocada de los objetos, de la región HII y de los flujos que colisionan en ambas regiones de la nebulosa. \\

De acuerdo a lo dicho anteriormente, vemos varios componentes involucrados en la formación de los arcos de proa en la Nebulosa de Orión, por tanto en este primer capítulo exploraremos un poco sobre la formación estelar en Orión, la población de estrellas masivas en esta región y su papel e importancia en la formación de los choques y la subsecuente formación de estrellas. Se hablan de los proplyds y su naturaleza, de sus flujos de gas ionizados provenientes de su frente de ionización y de la interacción de este con el viento estelar para formar los choques. Por último, se menciona en detalle la naturaleza de los arcos de emisión, donde se distinguen dos grupos; un primer grupo, que corresponde a los choques de proa de los clásicos proplyds y un segundo grupo, que son los arcos hiperbólicos LL en las regiones más alejadas de la nebulosa, donde se sugiere que los choques se forman por la interacción de un viento de una estrella T-Tauri con el transónico flujo de champaña del núcleo de la nebulosa. En los demás capítulos de la tesis estas suposiciones son probadas.                

\section{Orión y la formación estelar}
\label{sec:formation} 


Orión es la región de formación estelar más estudiada, debido a que las estrellas jóvenes y el gas nos proporcionan señales claras sobre la física en los procesos de formación estelar, la formación, evolución  y destrucción de las nubes en las que se forman las estrellas, además nos dan pistas claves y sutiles de la dinámica del medio interestelar y del papel que cumplen las estrellas de alta masa y las asociaciones OB en los ciclos del gas entre las distintas fases del medio interestelar.\\

\subsection{Estrellas masivas en Orión}
\label{sec:star}

Primero hablemos de la \textbf{Vecindad Solar} que es un caso particular donde es posible el estudio de los movimientos y distribuciones de las estrellas jóvenes relaciondas con el gas, que nos permiten trazar la historia de la formación estelar y la del medio interestelar. Como hay estrellas masivas, se crean regiones H II y estas junto a estrellas T-Tauri de baja masa trazan los sitios más recientes de formación estelar con edades entre 3 y 5 Myr \citep{Bally:2008a}, de la misma manera las asociaciones OB pueden trazar la historia de la formación estelar. Las asociaciones OB son las estrellas más masivas de la región, con ellas se pueden identificar lugares donde se han formado estrellas hace 40 Myr. Es así que las posiciones, velocidades, edades y masas de estrellas jóvenes y las propiedades del gas relacionadas con las asociaciones OB, son claves para entender la historia de la formación de estrellas y el origen, evolución y destrucción de las nubes moleculares en los últimos 100 años, logrando con esto desentrañar la naturaleza y la reciente historia del medio interestelar vinculada al nacimiento de las estrellas en esta región de la Via Láctea.\\

 Como el objetivo de nuestro estudio está centrado en la formación de estrellas jóvenes de baja masa, asociadas a estrellas masivas y regiones HII particularmente en \textbf{Orión}, entonces nos concentraremos en esta región. Las asociaciones OB en Orión consiste en un grupo de estrellas de diferentes edades que están parcialmente superpuestas a lo largo de nuestra línea de visión, dentro de estas región existen varios de estos subgrupos OB integradas por este tipo de estrellas masivas. Por ejemplo tenemos el grupo OB1a, del cúal muchos trabajos coinciden de que se trata del grupo de estrellas masivas más viejo de esta región, se encuentra ubicado en el noroesta del \textit{Cinturón de Orión} y su edad oscila entre 8 y 12 Myr \citep{Blaauw:1991, Brown:1994} dentro de este grupo hay un subgrupo conocido como el grupo 25 Orionis (25 Orionis group en inglés). El subgrupo OB1b, está centrado en el cinturón y se ha estimado que su edad comprende un rango entre 1.7 y 8 Myr, esto último es incosistente con la presencia de las tres estrellas gigantes (\(\zeta\) Orionis, \(\epsilon\) Orionis y \(\delta\) Orionis) puesto que deberían ser al menos 5 Myr más viejas, deacuerdo lo que dicen sus masas \citep{Bally:2008a}.\\

\begin{figure}
  \centering
  \includegraphics[width=.95\linewidth]{figuras-tesis/Orion-OB.jpg}
  \caption{Campo amplio de la región de Orión, que muestra los diferentes sub-grupos formados por las asociaciones OB. Tomado de wikipedia.org.}
  \label{fig:orionOB}
\end{figure}
  
En esta región se encuentra el subgrupo OB1c con edades entre 2 a 6 Myr, consiste básicamente  en estrellas que se encuentran en \textit{Orion's Sword} por su nombre en inglés, justamente frente de la Nebulosa de Orión . Este subgrupo contiene dos cúmulos, NGC 1980 ubicado en el extremo sur de \textit{Sword} y NGC 1981 situada en el extremo norte (ver figura \ref{fig:OB1-association}). Las estrellas más viejas en el Sword se superponen con poblaciones de estrellas mucho más jóvenes asociadas a la Nebulosa de Orión, M43, NGC 1977, OMC1 y 3 regiones en el \textit{Integral Shaped Filament} en el extremo norte de la nube molecular Orión A. Por otro lado tenemos a OB1d, que está formado por las estrellas de el cúmulo de la Nebulosa de Orión (ONC por sus siglas en inglés) situada en la nube molecular Orión A y por NGC 2024 ubicada en la nube molecular Orión B, son los dos cúmulos más grandes de este grupo que además resulta ser jóven, es decir con edades que van desde 2 Myr \citep{Muench:2008a}. Es difícil separar estos dos tipos de poblaciones estelares, puesto que no es claro aún si estos dos subgrupos (1c y 1d) representan diferentes poblaciones o más bien son grupos estelares jóvenes y viejos que se formaron en la nube Orión A en diferentes épocas, que desde luego han emigrado.

\subsection{Formación estelar en Orión}
\label{sec:frormacion}

Es bien sabido que la formación estelar ocurre cuando grandes nubes moleculares como Orión colapsan debido a su propia gravedad, es en ese momento cuando se desencadena la formación de estrellas de alta y baja masa. Como ya se dijo; en Orión tales fenómenos están presentes. Por ejemplo el subgrupo OB1d además de contener; el cúmulo de la nebulosa de Orión (ONC) y a NGC 2024 como se describió arriba, contiene  una docena de pequeños cúmulos y una distribución de estrellas en el fondo que están mas o menos aisladas, que se han formado en núcleos a lo largo de nubes moleculares en Orión, es el caso de las estrellas formadas en NGC 2068 y NGC 2071 en Orión. Varios miles de estrellas en su mayoría de baja masa, miembros del subgrupo 1d fueron formadas a partir de la \textit{Integral Shaped Filament} \citep{Bally:1987} en la parte Norte de la nube molecular Orión A, que contiene como ya sabemos a la Nebulosa de Orión \citep{Johnstone:1999}. Entonces cerca de 2000 estrellas de baja masa con edades menores a \(10^6\) años están concentradas alrededor de un cúmulo de estrellas masivas denominado el Trapecio en la misma Nebulosa de Orión \citep{Hillenbrand:1997} ubicada esta última a una distancia de 436~\(\pm\)~20~pc \citep{Odell:2008a}. Hay que subrayar que  cientos se están formando en el núcleo denso de OMC2 y en 3 núcleos situados en la parte norte de la Nebulosa de Orión. Hay que decir, que hay otro tipo de objetos que se han formado bajo estas circunstancias (proplyds y objetos LL) de los cuales hablaremos más adelante. A pesar de que no se sabe con certeza acerca de todos los miembros de la asociación OB, es probable que entre 5000 y 20000 estrellas se han formado en la región de orión en los últimos 15 Myr. No obstante las edades y ubicaciones de varios subgrupos en Orión indican que la formación de estrellas, ha sido propagada a través de de la nube de Orión de una forma secuencial \citep{Bally:2008a}. \\  
 
Ya desde hace muchos años se tiene conocimiento de que las estrellas fugitivas son comunmente las estrellas O, raramente se da entre estrellas B y es inexistente en las estrellas de tipo espectral posteriores a las ya mencionadas \citep{Gies:1986, Gies:1987}, en este sentido Orión es una fuente de varias estrellas fugitivas, dentro de esta se incluyen a AE~Auriga que tiene una velocidad de \(150~\text{km}~\text{s}^{-1}\)  y a \(\mu\)~Columbae con una velocidad de \(117~\text{km}~\text{s}^{-1}\) moviendose exactamente en la dirección opuesta \citep{Blaauw:1991}. Ahora datos de Hipparcos sobre movimientos propios han mostrado que estas dos estrellas y la colisión de vientos de la binaria de rayos-x \(\iota\)~Orionis, estaban ubicadas en la misma posición en el cielo hace más o menos 2.6 Myr \citep{Hoogerwerf:2001}, algunos científicos han argumentado que estas dos estrellas (AE~Auriga y  \(\mu\)~Columbae) junto a \(\iota\)~Orionis experimentaron una interacción de cuatro cuerpos que los llevó a sufrir cambios significativos, de tal manera que las estrellas más masivas se volvieron la binaria \(\iota\)~Orionis, mientras que para las estrellas menos masivas la suerte fue otra, puesto que la energía gravitacional liberada durante el encuentro lanzó a estas dos estrella fuera de la región a muy altas velocidades \citep{Gualandris:2004}.\\   

Tenemos que los movimientos propios en el cúmulo de la Nebulosa de Orión la sitúan en la interacción de los cuatro cuerpos aproximadamente, hace 2.6 Myr. Es así que la presencia de algunas de las estrellas viejas en el cúmulo de la Nebulosa de Orión; advierten que ha ocurrido formación estelar en esta región, sin embargo el número de estrellas viejas indican que la formación estelar en el gas para formar la ONC era más suave hace 2.6 Myr. No obstante, hay que resaltar que la tasa de formación estelar se ha ido acelerando con el tiempo, culminando recientemente con la formación de un grupo de estrellas masivas conocidas como el Trapecio y este proceso aún continua hasta el día de hoy.\\

\begin{figure}
  \centering
  \includegraphics[width=.95\linewidth]{figuras-tesis/OB1-association.jpg}
  \caption{Parte sur de  Orión que contiene los subgrupos OB1c y OB1d de las asociaciones OB1 de Orión. También se logra apreciar que el subgrupo OB1c parece estar directamente en frente del subgrupo más jóven OB1d \citep{Bally:2008a}. Otros cúmulos son marcados.}
  \label{fig:OB1-association}
\end{figure}
  

Si se asume que NGC 1980\footnote{NGC 1980 ha sido asociado con el subgrupo 1c de la asociación OB ubicada justamente en frente de la Nebulosa de Orión y también con la nube molecular Orión A (ver figura \ref{fig:OB1-association}).} comparte su movimiento através del espacio con \(\iota\)~Orionis, este cúmulo podría haber estado situado en el mismo lugar que el material, del cual más tarde se formaría la Nebulosa de Orión. Esto lleva a pensar que el material del cual se formaron la Nebulosa de Orión y la ONC estaba aparentemente cerca de NGC 1980 hace varios millones de años, sugiriendo que la formación de las mismas estaría desencadenado por la estrellas viejas del cúmulo NGC 1980. Ahora si esto es verdad, las estrellas más viejas de la ONC pueden ser mienbros de NGC 1980 y del sub-grupo OB1c.\\

\citet{Bally:2008a} dice que las ubicaciones y las edades de los grupos estelares en Orión, indican que la formación estelar puede propagarse a través de una nube de una forma no lineal. No obstante, una primera generación de estrellas desencadena el nacimiento de las posteriores generaciones. Es así que en Orión al parecer el subgrupo 1a fueron las primeras estrellas en formarse, consecuentemente estas estrella masiva contribuyeron al nacimiento de las 25~Ori y del subgrupo 1b. Posteriormente estas activaron la formación estelar en \textit{Orion's Sword} al sur, \(\sigma\)~Ori en el sureste, y posiblemente \(\lambda\)~Ori en el norte, por tanto en los últimos Myr se ha propagado la formación estelar dentro de \textit{Filament Integral Shaped} en la nube molecular Orión A, para formar la Nebulosa de Orión, M43, OMC2, OMC3 y NGC 1977.\\ 

\subsection{El papel de las estrellas masivas en la formación estelar}
\label{sec:star-masivas}

Las estrellas masivas inyectan energía en el medio interestelar a través de su radiación del continuo de Lyman (EUV), de sus vientos estelares y a través de la explosión de supernovas (SN). Ahora, si se usa la función de masa estandar (IMF), entonces la población estimada de estrellas jóvenes en Orión indica que entre 30 y 100 estrellas más masivas que 8 M\(\odot\), se han formado en esta región en los últimos 12 Myr, muchas de estas estrella han alcanzado la secuencia principal y posteriormente han explotado. Usando la relación edad-masa \(\tau(\text{M})~\propto~\text{M}^{-\beta}\), con \(\beta = 1.6 \pm 0.15\) en un rango de masas de 8 a 80 \(\text{M}_{\odot}\) \citep{Shull:1995}, ha mostrado que estrellas en el subgrupo 1a más masivas que 13 \(\text{M}_{\odot}\) han explotado. En los subgrupos 1b y 1c, estrellas con edades promedio de 6 Myr y con masas mayores a 20 \(\text{M}_{\odot}\), todo parece indicar que también han corrido con la misma suerte. Es así que han habido entre 10 y 20 explosiones de supernovas en la región de Orión en los últimos 12 Myr. Como consecuencia esta energía cinética liberada (\(>10^{52}~\text{ergs}\)) ha formado una enorme burbuja de gas de emisión de rayos-x, que se ha extendido creando una cáscara masiva de gas y polvo, conocida como la superburbuja Orión/Eridanus.\\

La estructura del medio interestelar en la  burbuja Orion/Eridanus proporcionan evidencias de que la energía liberada por estrellas de alta masa ha alterado profundamente la cinemática, la forma y la estructura del gas en esta región, debido a que la radiación de las asociaciones OB han probocado que la burbuja se haya inflado un poco más. La emisión de \ha{} trasa la ubicación actual del frente de ionización en Orión, además estas bajas densidades del gas se están expandiendo con una velocidad promedio cerca de 10 a 60 Km~\(\text{s}^{-1}\) hacia altas latitudes galáctica y hacia nosotros, este gas se puede ver en absorción y especialmente en el UV.\\
         
Por otro lado si somos más exigentes y nos situamos en determinadas regiones, por ejemplo donde se situa a la ONC, tenemos que la ionización está dominada por  \(\theta^1\ \text{Ori}\ \text{C}\) que es una de las estrellas masivas y jóvenes que forman el ya mencionado Trapecio. La densidad del gas ionizado decrece desde un pico de densidad electrónica de unos  \(10^{4}~\text{cm}^{-3}\) en el frente de ionización, puesto que el gas se acelera lejos del frente \citep{Henney:2005a}. En la parte más brillante de la nebulosa es decir, en el oeste del Trapecio la capa que emite es delgada (\(< 0.05 ~\text{Pc}\)), mientras que la región que emite en el este de la nebulosa es mucho más gruesa (\(\simeq 0.3~\text{Pc}\)), y esto puede relacionarse fácilmente con la extensión lateral y la distancia de  \(\theta^1\ \text{Ori}\ \text{C}\) al frente de ionización. Por otro lado los choques estacionarios que se forman en el frente de los Proplyds cerca de la estrella ionizante \citep{Bally:2000a}, son un indicativo de que hay una cavidad formada por los vientos de alta velocidad que vienen de esta estrella luminosa, además  la presencia de líneas de He I en absorción en el espectro de las estrellas del Trapecio \citep{Odell:1993, Baldwin:1991} indican que hay baja densidad en las regiones que se encuentran en la vencidad del centro del cúmulo de la Nebulosa de Orión. \\

\textbf{En resumidas cuentas}, se tiene que la presencia de estrellas másivas dispara el nacimiento de futuras generaciones de estrellas, como ha ocurrido en la región de Orión. Por otro lado hemos aprendido que los vientos estelales que son básicamente un flujo de partículas cargadas, que vienen de las estrella másivas OB, o en el caso particular de \(\theta^1\ \text{Ori}\ \text{C}\) del grupo del Trapecio en el Cúmulo de la Nebulosa de Orión crean zonas de baja densidad. Además estos vientos estelares interaccionan con el gas de la Nebulosa para formar las ondas de choques, también se da el caso que chocan con otros flujos de gas provenientes de los proplyds en las cercanías de la estrella ionizadora formando los ya mencionados choques estacionarios y con estrellas T Tauri según sea el caso en las partes más alejadas para formar los arcos de emisión. Dichos choques compactan el gas en la nebulosa y crean densidades no homogéneas que provocan el colapso gravitacional de la nube.\\

\section{Proplyds}
\label{sec:proplyds}

Hasta el momento ya es de nuestro conocimiento, que la Nebulosa de Orión alberga un conjunto de \textit{Objetos Estelares Jóvenes} (YSOs por sus siglas en inglés). Por tanto esta nebulosa nos brinda la posiblidad de estudiar las estrellas, que aún están siendo rodeadas por su material primordial. En primera instancia una oportunidad de estudiarlas surge cuando estos objetos son iluminados por una estrella ionizante, en este contexto  por \(\theta^1\ \text{Ori}\ \text{C}\), de tal manera que el gas que rodea a las estrellas será ionizado y como consecuencia este material será visible en las mismas líneas de emisión que la nebulosa. Por otro lado es posible ver la componente del polvo en extinción contra la emisión de la nebulosa, dado que la mayoría de la emisión de la nebulosa viene del fondo como ha argumentado \citet{Odell:2008}. Pero no todo termina aqui, puesto que la iluminación de la estrella masiva sobre estos YSOs, no sólo permite verlos, sino que además en el proceso parcial o total de la fotoionización de los mismos, genera un excedente de presión que provoca que el material sea expulsado a través del mecanismo de la fotoevaporación y en este sentido habrá como resultado una destrucción inevitable de sus envolventes. Estas estrellas jóvenes que tienen esa particularidad, es decir caracteríticas especiales locales de su entorno, se han etiquetado como una subclase dentro de los YSOs y han sido llamados proplyds (un acrónimo para ``disco protoplanetario'') \citep{Odell:1994}.\\

\begin{figure}
  \centering
  \includegraphics[width=.95\linewidth]{figuras-tesis/proplyd_182-413.pdf}
  \caption{Imagen de los proplyds 182-413 (izquierda) y 183-419 (derecha). Son observaciones del \textit{Telescopio Espacial Hubble} (\textit{HST}). La imagen de arriba corresponde a emisión de \(\mathrm{[O\,I]}~6300~\text{\AA{}}\)  y la de abajo a emisión de \(\ha\). En ambos proplyds sobresale lo que parece ser un disco circunestelar muy brillante en  \(\mathrm{[O\,I]}\) \citep{Bally:2000a}. La emisión de \(\mathrm{[O\,I]}\) cerca del frente de ionización es producido por la excitación por colisión. El campo de visión en cada marco es de \(4.55'' \times 13.65''\).}
  \label{fig:182-413}
\end{figure}
  
Siendo más rigurosos un proplyd puede ser definido como una estrella de baja masa presecuencia principal, envuelta en un  disco protoplanetario que esta siendo fotoevaporado por los fotones ultravioletas (UV) de una estrella masiva. En la Nebulosa de Orión los proplyds son vistos en líneas de emisión como se dijo arriba, con una una forma alargada donde uno de sus extremos que es más ancho que el otro (cabeza del proplyd), apunta en dirección a la estrella  \(\theta^1\ \text{Ori}\ \text{C}\). 

\subsection{ Descubrimiento}
\label{sec:descubrimiento-proplyds}

El primero de estos objetos descubierto y posteriormente identificado como proplyd fue LV 2 (167-317) y fue visto en la cercanías del Trapecio. En trabajos posteriores se idenficaron un conjunto de seis líneas de emisión no resueltas, es decir estrellas; en la las cercanías del Trapecio \citep{Laques:1979}. Hasta el momento no se conocía de manera clara la naturaleza de los proplyds, pero dado que son fuentes de radio compactas de emisión térmica, fueron descubiertos en estudios llevados a cabo por el \textit{Very Large Array} realizadas en el interior de la región de Huygens \citep{Garay:1987}, por tanto con las interpretaciones de las radio fuentes, dió paso  hacer una correcta identificación de los mismos \citep{Churchwell:1987}. Con la intervención del \textit{Telescopio Espacial Hubble} (\textit{HST}) se pudo establecer la verdadera naturaleza de los proplyds, ya que con la cámara WFPC2 \citep{Odell:1994} se pudieron obtener imágenes más claras y puras de esta región, como se puede ver en la figura \ref{fig:proplyd-hst}. En dichas imágenes se pueden apreciar unos objetos cerca de la estrella ionizadora, con una estrella central de baja masa y en algunas ocasiones con una región oscura en el centro.\\ 

\begin{figure}
  \centering
  \includegraphics[width=.95\linewidth]{figuras-tesis/proplyd-HST-WFPC2}
  \caption{Imagen tomada con la Cámara Planetaria (PC) WFPC2-\textit{HST}. Los colores indican las líneas de emisión o filtros usados;  verde = \(\ha~\lambda6563\), rojo =  \(\nii~\lambda6584\) y  azul = \(\oiii~\lambda5007\) \citep{Bally:1998a}. En la imagen son perceptibles los proplyds 183-405 y 182-413, además se observa un objeto con emisión algo débil llamado 183-419.}
  \label{fig:proplyd-hst}
\end{figure}
  
Como los proplyds emiten radiación, en las mismas líneas que la nebulosa, se obtuvo su espectro  corregido por la contribución de la radiación provenientes del fondo. Entonces con las primeras observaciones terrestres tomadas con el Fabry-Perot \citep{Fuente:2003} y del espectrómetro Manchester Echelle Spectrometer del telescopio Isaac Newton \citep{Henney:1997} se lograron obtener espectros muy útiles, pero no se tenía mucha confianza en estos espectros debido a la alta correción que se había hecho. Más tarde con espectrografía echelle de telescopios terrestres más grandes se produjeron mejores espectros a partir de líneas más fuertes de cuatro proplyds: 170-337 (HST 2), 177-341 (HST 1), 182-413 (HST 10) y 244-440 \citep{Henney:1999a}.  Posteriormente se obtuvieron observaciones espectroscópicas con una alta resolución espacial del proplyd 167-317 (LV 2) \citep{Vasconcelos:2005}. Los objetos 158-327 (LV 6) y 159-350 (HST 3) fueron estudiados con una buena resolución espectral en unas imágenes de la Faint Object Camera por su nombre en inglés del \textit{HST} \citep{Bally:1998a}. No obstante el más completo estudio realizado sobre  LV 2  a una alta resolución, vienen de unas observaciones \citep{Henney:2002a} del doblete a 1907-1909~\A{} de \ciii{}, usando el espectrómetro del \textit{HST/STIS}, este análisis estuvo centrado en el microjet que sale del frente de ionización del proplyd.    

\subsection{Modelo estandar de los proplyds}
\label{sec:modelo}

En la sección anterior se ha dicho que la verdadera naturaleza de los proplyds en la ONC fue revelada por observaciones de alta resolución del \textit{HST}, aunque ya anteriormente \citet{Churchwell:1987} había escudriñado en su naturaleza con observaciones en el radio. Al comienzo se pensó que la cabeza del proplyd, se formaba por la interacción de un viento suave proveniente del disco de acreción estelar, con el viento rápido de \(\theta^1\ \text{Ori}\ \text{C}\). Como a veces suele suceder en la ciencia, este modelo se dejó a un lado, puesto que se estableció que esta parte brillante del propyid (cabeza), no eran más que frentes de ionización locales \citep{Odell:1994}, en los cuales su brillo superfial disminuía como es de esperarse con la distancia a  \(\theta^1\ \text{Ori}\ \text{C}\).\\

\begin{figure}
  \centering
  \includegraphics[width=.95\linewidth]{figuras-tesis/model-proplyds.pdf}
  \caption{Modelo ampliamente aceptado para los proplyds, que básicamente representa un flujo fotoevaporado. Se tiene que el disco proptoplanetario de una estrella jóven de baja masa es afectado por los fotones estelares FUV y EUV. En este sentido la radiación FUV penetra en la superficie del disco de acreción generando una tasa de pérdida de masa, implicando con ello que se forme un flujo lento de gas neutro. Este gas neutro funciona como una capa protectora, pues absorve la radiación EUV y es así como se forma un frente de ionización que apunta en la dirección de la estrella ionizadora, entonces este frente de ionización no es más que la cabeza del proplyd. La cola del proplyd se forma debido a la fotoevopación de la parte trasera  del disco, por la radiación difusa UV. No obstante el campo de ionización difuso, que es el resultado de la recombinación del hidrógeno a su estado base \citep{Henney:1999a}, influyen de manera importante en el flujo ionizado de la cola. Claramente se puede ver que un proplyd es semejante a un cometa, aunque su cola no se debe al arrastrado por la presión de radiación como ocurre en los cometas reales. }
  \label{fig:modelo-proplyd}
\end{figure}

Es así que el modelo estandar y ampliamente aceptado postula la existecia de un  disco de acreción interno de material molecular que está rodeando una estrella jóven de baja masa presecuencia principal, que a su vez está siendo fotoevaporado por los fotones ultravioletas de una estrella masiva \citep{Johnstone:1998, Henney:1998a}. Este disco sólo es afectado por la radiación externa Ultravioleta Lejano (FUV, \(\lambda > 912~\text{\AA{}} \)), es decir radiación con energía menor a los 13.6eV necesarios para fotoionizar el hidrógeno (ver figura \ref{fig:modelo-proplyd}). Esto ocurre porque la fotodisociación de gas molecular que es calentado y suavemente expulzado a partes externas del disco forma una atmósfera extendida que es ópticamente gruesa a la radiación del continuo Lyman (EUV, \(h\nu > 13.6~\text{eV}\); \(\lambda < 912~\text{\AA{}} \)), en otras palabras los fotones FUV son los responsables de disociar las moléculas y de calentar el gas de la región de fotodisociación (PDR) a \(T\sim100-1000~\K\) dejando como resultado una estela de material neutro \citep{Johnstone:1998}. Lo anteriormente dicho implica que esta atmósfera interna está siendo rodeada por un frente de ionización local, que es más brillante en la dirección en la que se encuentra la estrella ionizante dominante, también tiene una zona con un brillo más débil que tiene la forma de la cola de un cometa, debido a la fotoinización del material por la radiación difusa del continuo de Lyman.\\ 

\subsection{Choques de proa asociados a proplyds}
\label{sec:shock-proply}

\begin{figure}
  \centering
   \includegraphics[width=.95\linewidth]{figuras-tesis/shock-proplyd}          
  \caption{Región de la Nebulosa de Orión centrada en el Trapecio, como el resultado de un mosaico de imágenes de la Cámara Planetaria; WFPC2 del \textit{HST} tomadas durante el ciclo 4 en unas observaciones de la Nebulosa de Orión, usando unos filtros de banda angosta. Los colores indican; verde es \ha{}, rojo es \nii{} y azul es \oi{}. Imagen tomada del artículo de \citet{Bally:1998a}.}
  \label{fig:shock-proplyd}
\end{figure}

Los proplyds que están situados cerca de la estrella masiva \thC{} están acompañados por arcos débiles, que son visibles por la emisión de \oiii{} y de \ha{}. También se pueden ver por la emisión de polvo de silicato, es decir en el infrarrojo a 11.7~\(\mu \text{m}\) \citep{Arredondo:2001, Hayward:1994}. A partir de unas imágenes de \ha{}+\nii{}  del \textit{HST} donde se observan los arcos en estas líneas de emisión, \citet{Bally:1998a} propone que estos arcos trazan la estructura de choques de proa formados por la interacción del viento estelar proveniente de \thC{}, con el flujo de material fotoevaporado del frente de ionización de los propyds. La figura~\ref{fig:shock-proplyd} es una región de la Nebulosa de Orión centrada en el Trapecio, que contiene los primeros objetos descubiertos por \citet{Laques:1979}. Cerca de una docena de proplyds iluminados externamente son visibles en esta imagen, además se logra apreciar que algunos tienen arcos radiativos y que todos tienen un frente muy brillante por las líneas de emisión en dirección a \thC{}.         

\section{Objetos LL  en la  Nebulosa de Orión}
\label{sec:objeto-ll}

Los típicos Objetos LL llamados así por la primera versión de estrellas LL orionis descubierta en la Nebulosa de Orión, son  básicamente arcos de emisión (vistos en líneas de emisión) asociados a estrellas jóvenes de baja masa presecuencia principal situadas en regiones externas de la Nebulosa de Orión \citep{Henney:2013a} (ver figura \ref{fig:LL1}). El prototipo de estos objetos es la estrella T Tauri LL~Ori cuyo arco de emisión fue descubierto hace 36 años \citep{Gull:1979a}. Posteriormente se identificaron seis objetos más, con características similares \citep{Bally:2001a} y fueron denotados empezando desde LL1 hasta LL7, donde el primero de estos corresponde a LL~Ori. La lista de Objetos LL detectados siguió en aumento, puesto que gracias a datos de líneas de emisión en el óptico (\(\ha~6563~\text{\AA{}}\), \(\nii~6584~\text{\AA{}}\) y \([\text{SII}]~6716,6731~\text{\AA{}}\)) del Telescopio Espacial Huble (HST) \citep{Bally:2000a, Bally:2006a} se han identificado cerca de 20 objetos. No obstante muchos de estos objetos tienen jets muy coliminados que se originan en la estrella T Tauri, que de alguno u otra forma alteran la morfología de los arcos de emisión. Es el caso de LL1 quien posé un jet hipersónico del tipo Herbig Haro conocido como HH 888.\\

\begin{figure}
  \centering
  \includegraphics[width=.95\linewidth]{Figures/ll-figs/LL1-schematic.pdf}
  \caption{\textit{Izquierda}. Ubicación de LL~Ori en el suroeste de la Nebulosa de Orión. Esta imagen es el resultado de la convinación de observaciones del HST-WFPC2 \citep{Odell:1996} con los filtros: \(\ha~\lambda6563\) (verde), \(\nii~\lambda6584\) (rojo) y \(\oiii~\lambda5007\) (azul). Hay una región de saturación en el Trapecio y esta aparece en color blanco siendo una imagen superpuesta de \(\ha\). \textit{Derecha}. Es una ampliación de la zona donde se encuentra LL~Ori. Se puede ver el hiperbólico choque de proa, que se forma debido a la interacción de un viento de una estrella T Tauri, con  el flujo de champaña de gas ionizado  proveniente del núcleo de la Nebulosa de Orión. También se puede apreciar un objeto HH (jet) asociada a la  estrella T Tauri, en  el cual el choque de este tiene movimientos propios \citep{Henney:2013a} y una velocidad radial (flecha de color). Las velocidad de las alas del choque es de \(\sim 20 ~\text{km}~\text{s}^{-1}\), mientras que la velocidad del jet es muy superior (\(60-120 ~\text{km}~\text{s}^{-1}\)). Imagen de William Henney.}
  \label{fig:LL1}
\end{figure}

\subsection{Naturaleza de los arcos de proa LL }
\label{sec:choques}

Los objetos LL Orionis pueden ser interpretados como la interacción supersónica de un viento interno de una estrella T-Tauri, con el flujo ambiental de la Nebulosa, aunque es apresurado pensar en una típica estrella T Tauri como el objeto que proporciona el viento interno, puesto que aún no se conoce con certeza la naturaleza del viento interno, debido a que el flujo fotoevaporado proveniente del disco protoplanetario de un proplyd también es un candidato muy fuerte para el viento interno (en esta tesis se intentarán aclarecer estós argumentos). En una escala más grande el viento externo parece ser originario de la  región H II en el núcleo de la Nebulosa de Orión. Esto es debido a que cuando un frente de ionización envuelve un objeto muy denso, se forma un flujo fotoevaporado de gas ionizado con un frente D-crítico \citep{Dyson:1968}. Entonces, a este flujo de ahora en adelante lo llamaremos flujo de champaña y como han indicado \citet{Mellema:2006, Arthur:2011, Ercolano:2012}, este flujo surge durante los procesos de evolución de una región H II dentro de una nube molecular turbulenta.\\

\paragraph{Esquema general de los Objetos LL.} La figura \ref{fig:esquema-arcos} nos muestra un esquema general de los choques de proa en la Nebulosa de Orión, es así que los arcos hiperbólicos de los objetos LL se forman, debido a que el flujo de Champaña (izquierda), que entre otras cosas es ligeramente supersónico (\(M\simeq2\)) choca con un obstáculo (derecha), que en este caso es un flujo también supersónico asociado a una estrella jóven de baja masa, no obstante el choque externo es muy radiativo proporcionando con esto, un arco de emisión brillante, es decir muy visible. La naturaleza del obstáculo aún no es clara; puesto que  es posible que sea el flujo suave de un gas ionizado proveniente del frente de ionización del Proplyd (arriba) o podría ser el viento de una estrella T Tauri (abajo).\\ 

\begin{figure}
  \centering
  \includegraphics[width=.7\linewidth]{Figures/ll-figs/LL-outer-inner.jpg}
  \caption{Esquema general de los Objetos LL}
  \label{fig:esquema-arcos}
\end{figure}

\section{Choques de proa estacionarios en la Nebulosa de Orión}
\label{sec:shocks}

\paragraph{La zona chocada.} En general se dice que los choques de proa de los objetos LL y de los proplyds son estacionarios, debido a que no se les han detectado movimientos propios y además muestran pequeñas velocidades radiales. Por otro lado, la región chocada es ancha debido a que la interección de dos viento, provoca que se forme un doble choque, entonces tenemos un arco interno y un arco externo separados cierta anchura. Ahora sí unos de estos choques es fuertemente radiativo será visible un arco de emisión. Si este el caso, el choque podría considerarse isotérmico, puesto que la zona de enfriamiento en el choque es muy pequeña cuando el flujo es medianamente supersónico (\(20-60~\text{km}~\text{s}^{-1}\)) y la densidad es alta (mayor a unos cientos \(\cm^{-3}\)). Este fenómeno es comparable a lo descrito por \citet{Henney:2002} en la interacción de los vientos de dos proplyds; cuando la temperatura del gas en la cáscara chocada se eleva por la termalización de la energía cinética pre-choque, dando como resultado un aumento en las emisiones, puesto que la energía térmica se irradia y el gas retorna nuevamente a su estado de equilibrio. En el caso particular de LL1, la emisión en la cáscara chocada en equilibrio está dominada por líneas de recombinación tales como \ha{}, mientras que para las líneas excitadas colisionalmente dominan las de \oiii{} en los objetos cerca del Trapecio y las líneas de \nii{} en los arcos LL donde las líneas de \oiii{} son débiles.\\ 

\begin{figure}
  \centering
  \includegraphics[width=.95\linewidth]{Figures/ll-figs/wind-geometry-extended-lores.jpg}
  \caption{Esquema de la interacción de los vientos en la Nebulosa de Orión. la región de color verde corresponde a gas molecular neutro cuya temperatura oscila entre los 50 y los \(1000~\K\); la zona roja es el gas fotoionizado, ahí la temperatura es de aproximadamente \(10^{4}~\K\) y las densidades van de \(10^{2} ~\text{a}~10^{4} ~\cm^{-3}\) y por último el color azul representa el material del viento estelar con \(T \geq 10^{6}~\K\) y \(n \sim 1~\cm^{-3}\). Las flechas hacen referencia al flujo de gas transónico y supersónico, por otro lado los choques de proa radiativos son ilustrados por las líneas rojas oscuras y gruesas, mientras que los choques no-radiativos son ilustrados por la línea discontinua azul y la línea de puntos también azul indica la discontinuidad de contacto. Los choques de proa ocurren en dos regiones de la nebulosa: una zona interna de interacción (cuadro izquierda arriba), donde los choques de proa externos se producen debido al viento hipersónico de una estrella másiva (\(V \sim 1000~\text{km}~\text{s}^{-1}\)) y una zona externa de interacción (cuadro derecha arriba), donde los choques externos se forman debido al flujo lijeramente supersónico y fotoevaporado de champaña (\(V \sim 20~\text{km}~\text{s}^{-1}\)). El choque externo es no-radiativo en la zona interna de interacción pero radiativo en la zona externa de interacción. Esto sugiere que el choque de proa interno es radiativo cuando el viento interno es un flujo de gas fotoevaporado de un proplyd. (Imagen de William Henney).}
  \label{fig:esquema-interraccion-vientos}
\end{figure}

\paragraph{Dos poblaciones de choques de proa.}
Los choques de este contexto se han divido en dos grupos; el primer grupo corresponde a los clásicos choques de proa de los proplyds \citep{Robberto:2005, Bally:1998a}, de estos objetos se ha habla con más detalle en \S\ref{sec:shock-proply} y el segundo grupo corresponde a los típicos arcos hiperbólicos LL (ver figura \ref{fig:esquema-interraccion-vientos}). En el caso de los proplyds que están situados en las proximidades del Trapecio, el choque de proa externo no es visible porque este se produce por la interacción con un viento muy rápido y de baja densidad (\(n \sim 1 ~\cm^{-3}\)) de una estrella O, dando como resultado que la cáscara externa sea no-radiativa. No obstante el arco interno si es visible, dado que el choque interno se forma a partir de un flujo de gas muy denso (\(10^{3}-10^{4}~\text{cm}^{-3}\)) y ligeramente supersónico (\(M \simeq 3\)) proveniente del frente de ionizacion del proplyd. En el caso de los arcos hiperbólicos, se tiene que están ubicados en regiones externas de la nebulosa, es decir están mucho más lejos del Trapecio, incurriendo en el hecho de que el arco exterior de la región chocada sea radiativo, por tanto en este domina la emisión. Otra característica diferenciadora de sus semejantes los proplyds, es que los arcos de estos tienden a hacer más abiertos.\\ 

\begin{figure}
  \centering
  \includegraphics[width=.95\linewidth, trim=30 20 30 30, clip]{figuras-tesis/proplyd-and-LL-arc.pdf}
  \caption{Tipos de choques estacionarios en la Nebulosa de Orión. (a) Choques de proas asociados a proplyds; el choque se forma por la interacción de un flujo de gas fotoevaporado que viene del proplyd a una velocidad entre 30-40 km \(\text{s}^{-1}\), con un viento que viaja a una alta velocidad (\(>1000 ~\text{km} ~\text{s}^{-1}\)) desde una estrella masiva del Trapecio. En la imágen aparecen los proplyds 177-341 y 173-342. También se observa un choque formado por la interacción de los vientos de dos proplyds.(b) Arcos hiperbólicos LL; el choque se forma debido a la interacción de un flujo de champaña de baja velocidad (\(\simeq 20~\text{km} ~\text{s}^{-1}\)) con un viento estelar que viene de una estrella T-Tauri o de un proplyd. Estas imágentes son tomadas de las observaciones del HST-ACS usando el filtro f658n, es decir de \(\ha + \nii\) y se observan los objetos; LL 6 con su respectivo jet HH 876, 308-3036 y 344-320. }
  \label{fig:proplydbow-objetoll}
\end{figure}

\subsection{Choques de proa producidos por Objetos HH y su relación con los Objetos LL}
\label{sec:herbig}

La importancia de hablar de los jets Herbig Haro en este trabajo radica en dos particularidades: primero, no confundir los choques de proa de los objetos LL con los choques de prao producidos por los jets HH. Se tiene que los choques de los objetos Herbig Haro a diferencia de los arcos LL, son producidos por la interacción de un jet colimado de material que se mueve a altas velocidades\footnote{Estos jets problablemente se originan en un objeto estelar jóven.}, con el ambiente de gas nebular \citep{Odell:1994}. Además de eso los choques HH muestran grandes movimientos propios y altas velocidades radiales. Segundo, varios objetos LL tienen jets muy colimados, lo cual a primera vista, sugiere que estos le dan la forma a los arcos LL, pero recientes estudios han mostrado que en el caso de LL1, la cinématica del choque de proa y del jet Herbig Haro (HH 888) no coincide, puesto que la del choque es simétrica y la del jet es asimétrica. Mientras para LL2 la asimetría cinématica del jet HH 505 es totalmente opuesta al del choque de proa del objeto LL en cuestión (Ver figura \ref{fig:LL1} y figura \ref{fig:objecthh} arriba izquierda). Esto parece indicar que los choques hiperbólicos son independientes de los jets Herbig Haro \citep{Henney:2013a}. Aunque también puede darse el caso de que las alas de los arcos pueden estar asociadas a los jets HH, como parece ser el caso de LL 6 en la figura \ref{fig:proplydbow-objetoll}, donde no se sabe si las alas de su arco son generadas por el jet o por el contrario no lo son.

\begin{figure}[htp]
\centering
\begin{tabular}{l l}
(a) & \\
 & \includegraphics[width=0.4\linewidth, trim=60 20 20 30, clip]{./Figures/ll-figs/LL2-Will-Crop-Annotate.jpg}
\includegraphics[width=0.5\linewidth, trim=10 10 10 10, clip]{./Figures/ll-figs/HH-529-10x10arcsec-annot.pdf}
\\
& \\[2\baselineskip]
(b) & \\
& \includegraphics[width=0.92\linewidth, trim=30 150 30 100, clip]{./figuras-tesis/hhbowshock.pdf}
\\
\end{tabular}
\caption{(\textit{a}) \textit{Izquierda}. En la imagen se observa otro de los ya estudiados objetos LL en Orión. Es básicamente la estrella T-Tauri IX Ori ubicada en una región lejana en el oeste de la Nebulosa de Orión, con su choque de proa asociado (LL1) y su jet bipolar HH 505. También son visibles regiones de emisión de \oiii{}. \textit{Derecha}. Superficie de trabajo producida por un jet hipersónico muy colimado conocido como HH 529. Tienen altas velocidades radiales (\(50-100 ~\text{km}~\text{s}^{-1}\)) y también altos  movimientos propios perpendiculares al arco (Imágenes de Henney). (\textit{b}) Choque de proa de un objeto HH conocido como HH 202, (izquierda) del HST-WFPC2 y (derecha) a 11.7 \(\mu\) (T-ReCS). Se tiene que en la imagen del HST los colores indican; verde \ha{}, rojo \nii{} y azul \oiii{} \citep{Smith:2005}.}\label{fig:objecthh}
\end{figure}

\subsection{Notación de los objetos LL y de los proplyds}
\label{sec:notacion}

Como ya se ha sujerido anteriormente, \citet{Odell:1994} estudiaron muy a fondo los proplyds y junto a  ello idearon una nomenclatura para nombralos, dicha notación está basada en la posición de la estrella en el cielo. En este sentido, las coordenadas en ascención recta y declinación de la estrella central son las piezas clave para la designación del proplyd. Si tenemos el proplyd con las coordenadas (A.R., DEC) = (5:35:17.67, -5:23:41.0), entonces este tendrá por nombre 177-341. En este orden de ideas la notación de los objetos LL y proplyds descritos en este estudio se basa en la configuración ya mencionada, además hemos utilizado este procedimiento para designar la nomenclatura de los arcos LL o choques de proa de proplyds que hemos identificados (nuevos) en la Nebulosa de Orión, que entre otras cosas 20 de estos objetos de 73 no han sido reportados previamente en la literatura.

\section{Estructura de la tesis}
\label{sec:estructura}

La estructura de la tesis está organizada de la siguiente forma. En el capítulo~\ref{chap:datos} presentamos una descripción de las observaciones empleadas en este trabajo, por tanto se discuten sobre las caracteríticas de las cámaras ACS y WFPC2 al bordo del \textit{HST}. En el capitulo~\ref{chap:methodology} se describen las mediciones de las formas de los choques de proa, las mediciones de los valores del brillo superficial y las calibraciones del flujo de \ha{} y \nii{}. En el capítulo~\ref{chap:theori} se muestra un tratamiento teórico de este trabajo, que son procedimientos matemáticos para describir los choques de proa estacionarios. En el capítulo~\ref{chap:results} presentamos los resultados de las modelaciones de las observaciones, donde se muestran las formas de los choques y demás parámetros obsevacionales, tambien se ilustran los resultados astrofísicos, en los cuales se muestra la naturaleza de las colisiones del viento de las estrellas masivas del Trapecio con el viento de estrellas T-Tauri o con el flujo de gas ionizado de  proplyds o de la interacción de estas últimas con el flujo de champaña del núcleo de la nebulosa en diferentes regiones de la Nebulosa de Orión. Por último, el capítulo~\ref{chap:conclu} es un compendio de las conclusiones a las que hemos llegado sobre el tratamiento de los choques de proa estacionarios, formados por la interacción de dos flujos.   
% \bibliography{luis-ref}

% \end{document}
