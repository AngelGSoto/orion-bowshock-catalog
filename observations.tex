% \documentclass{article}
% \usepackage[utf8]{inputenc}
% \usepackage{amsmath}
% \usepackage{natbib}
% \usepackage{graphicx}
% \usepackage{astrojournals} % Necesario para nombres de revistas en luis-ref.bib
% \usepackage[spanish, es-minimal]{babel}
% \bibliographystyle{apj}


% \title{Catalog of stationary bowshock arcs in the Orion Nebula}

% \begin{document}
% \maketitle

% \section{Observaciones y descricción de los datos }

\label{chap:datos}

Para este estudio se usaron observaciones del \textit{Telescopio Espacial Hubble} (\textit{HST}), debido a que este brinda las posibilidades de obtener imágenes concretas de la Nebulosa de Orión, permitiendo en este sentido examinar sus componentes estelares, como es el caso de los objetos LL que presentan fuertes líneas de emisión de \ha{} y \oiii{}. Donde se tiene que en las imágenes tomadas por el \textit{HST} los choques de proa resultan ser muy visibles en el óptco que en cualquier otras líneas de emisión, puesto que en la cáscara chocada domina la emisión de líneas de recombinación tales como \ha{} y para el caso de líneas generadas a patir de la excitación por colisión domina esencialmente \oiii{}. Por tanto en este trabajo se han usado las imágenes tomadas con la cámara \textit{Avanced Camera Survey} (ACS) y la cámara planetaria (PC por sus siglas en inglés) \textit{Wide Field Planetary Camera 2} (WFPC2). 

\section{Cámara  WFC-ACS , el programa GO-9825 y el filtro F658N }
\label{sec:acs}

En primer lugar  hemos trabajado con las imágenes de \citet{Bally:2006a}, las cuales son observaciones que han sido obtenidas con la cámara \textit{Wide Field Camera} (WFC) ACS abordo del \textit{HST}, durante el ciclo 12 del programa GO-9825, donde se obtuvieron 26 imágenes que cubren gran parte de la Nebulosa de Orión, cada imagen fue tomada durante una órbita del \textit{HST}. Las áreas cubierta fueron observadas usando el filtro F658N (\(\ha+\nii\)) y tiempos de exposición de 500 s por punto observado en la nebulosa. Dos posiciones, separadas por \(96''.8\) en el eje-y de los detectores del ACS fueron observados durante cada órbita \citep{Bally:2006a}. Se tiene que esta cámara está basada en un mosaico de dos detectores CCD de \(2046 \times 4096\) pixel, correspondiendo a un campo de visión de aproximadamente \(100'' \times 200'' \) para cada imágen resultante. En este sentido las distintas imágenes de la Nebulosa de Orión tomadas durante las 26 órbitas cubrieron un área total de aproximadamente 415 \(\text{arcmin}^{2}\). La figura \ref{fig:fields} muestra la ubicación de cada uno de los 26 campos (visitas) del ACS, donde se han superpuesto en una imagen de \sii{} tomada por el telescopio reflector Mayall ubicado en Kitt Peak cerca de Tucson, para mayor información hechar un vistaso al artículo de \citet{Bally:2001a}. Cada rectángulo representa el campo de cada visita del \textit{HST}.\\

\begin{figure}
  \centering
  \includegraphics[width=.8\linewidth, trim=0.7 0.7 0.7 30, clip]{figuras-tesis/fields-acs-sii.png}
  \caption{Posiciones de los 26 campos del ACS observadas con el \textit{HST} durante el ciclo 12 del programa GO-9825, superpuestas en una imagen de \(\sii\) obtenida con un mosaico de detectores CCD del telescopio reflector Mayall situado en Kitt Peak, Arizona. Los rectágulos representan las imágenes CCD del ACS con un tamaño de \(2046 \times 4096\) pixel cada uno. Esta imagen es tomada de \citet{Bally:2006a}.}
  \label{fig:fields}
\end{figure}

La importancia de utilizar las imágenes de Bally radica en que la señal a ruido es muy buena, debido a que tienen tiempos más largos de exposición, con esto se logró que el detector CCD recojiera el mayor número de fotones posibles por pixel, implicando que la señal sea muy alta en comparación al ruido.\\

Por otro lado, como ya se dijo arriba esta cámara (ACS) contiene el filtro F658N, el cual es un filtro de banda ancha (\(50 \text{Å}\)), que permite obtener imágenes de \ha{} + \nii{}, donde principalmente domina las líneas de \ha{} y en menor grado las líneas de \nii{}. Para nuestro trabajo vamos a utilizar estas imágenes para trazar los arcos radiativos de las estrellas LL ori, debido a que con este filtro es posible mapear el material que rodea a las estrellas con la más alta resolución posible y de la misma forma permite discriminar fuentes extendidas de emisión. Sin embargo, la desventaja de utilizar estas observaciones es que no tiene una cobertura completa de la Nebulosa de Orión\footnote{Es el caso de  LL7. Este objeto queda por fuera de los campos de Bally.} y además estas observaciones están contaminadas por líneas de \nii{}. Por último es importante señalar que de los 26 campos del ACS-F658N que mapean la región, nuestros objetos se encuentran repartidos en los campos: 01, 02, 06, 07, 08, 09, 14, 16, 17 y 24. En la figura \ref{fig:field-01} se muestra como es la apariencia de uno de estos campos (visita), que como ya se sabe es una imagen de \ha{} + \nii{}.\\

\begin{figure}
  \centering
  \includegraphics[width=.95\linewidth, trim=60 90 60 60, clip]{figuras-tesis/field-01-acs.pdf}
  \caption{Campo 01 del WFC-ACS; que corresponde a una imagen de \ha{}+\nii{} observadas con el \textit{HST} durante el ciclo 12 del programa GO-9825, usando el filtro de banda ancha F658N, en ella se puede apreciar el Trapecio y algunos choques de proa en sus cercanías.}
  \label{fig:field-01}
\end{figure}

\section{La cámara WFPC2 y el filtro F656N}
\label{sec:wfpc2}

Un compendio de imágenes tomadas con la cámara planetaria (PC) WFPC2, han de brindar otra oportunidad de estudiar los arcos LL de la Nebulosa de Orión. Puesto que la ya mencionada WFPC2, que entre otras cosa utiliza cuatro CCDs de $800\times800$ pixel a bordo del \textit{HST}, ha tomado un conjunto de imágenes que cubren en su mayoría a la Nebulosa de Orión a diferencia de las imágenes de Bally que no tienen una cubertura total de la misma. Se han usado 104 órbitas del \textit{HST} para dicho fin, como parte del \textit{Programa Treasury} \citep{Robberto:2013a}. En la figura \ref{fig:fields-robberto} se pueden apreciar los campos (visitas) del WFPC2, que entre otras cosas cubren un área total de 570.5 \(\text{arcmin}^{2}\).\\

\begin{figure}
  \centering
  \includegraphics[width=.95\linewidth]{figuras-tesis/fields-robberto.jpg}
  \caption{\textit{Izquierda}. Campos de WFPC2 superpuertos en una imagen de JHK de la Nebulosa de Orión de 2MASS. \textit{Derecha}. Representación de los campos de WFPC2 obtenidas a partir de 104 vistitas del \textit{HST}, como parte del \textit{Programa Treasury}. Mapean una área consirable de la Nebulosa de Orión. Las partes partes mas oscuras son las regiones de solapamiento entre las distintas visitas. También se puede ver la clasificación numérica de cada campo. Imagen tomada de \citet{Robberto:2013a}.}
  \label{fig:fields-robberto}
\end{figure}

Uno de los \textit{Filtros} usados en este ambicioso programa es el filtro F656N. Este es un filtro de banda angosta de \ha{}, que sólo permite el acceso de estas líneas (\ha{}~\(\lambda = 6563~\text{\AA{}}\)), es así que no permite que pasen las líneas de \nii{} (\(\lambda = 6583~\text{\AA{}}\)). Entonces para este trabajo usamos las observaciones de \citet{Robberto:2013a} (WFPC2) debido a que estas imágenes son sólo de \ha{}.\\

 En este sentido hemos de utilizar estos dos filtros porque de esta manera, será posible separar las emisiones de \ha{} de \nii{} y así poder utilizar las imagenes de ACS para determinar los parámetros observacioneles y astrofísicos de los choques estacionarios, puestos que estas tienen mejor resolución y así utilizar el brillo superficial de \ha{} para determinar el producto \(Mv\) de los LL~Ori.

\section{Otras observaciones de la cámara WFPC2:  viejos mosaicos}
\label{sec:mosaic}

Para este estudio también se cuenta con unas observaciones, un poco más viejas que mapean una región particular de la Nebulosa de Orión. Estas observaciones son el resultado de dos pragramas del \textit{HST}, el primero de estos es el programa GTO-5085 (Guaranteed Time Observer  por su nombre en inglés) cuyo principales investigadores fueron  \citet{Odell:1996} y el segundo corresponde al programa GO-5469 (General Observer por su nombre en inglés) con Jhon Bally como principal investigador. Las imágenes proporcionadas por el programa de Bally fueron usadas principalmente para el estudio de proplyds con una  alta resolución espacial, usando la cámara planetaria (PC) WFPC2 abordo del \textit{HST}. En estas imágenes el tamaño de lo pixeles es de \(0.0996''\) algo muy característico de las cámaras WF \citep{Holtzman:1995}. Los campos seleccionados para la formación de las imágenes se determinaron de tal manera que tengan una cobertura continua de un área en la parte central de la Nebulosa de Orión, en este sentido se tiene una superposición de los campos, que representados através de unos  masaicos de la cámara WFPC2, se hablan de varios mosaicos puesto que se utilizaron multpliples filtros del programa GTO-5085 para registrar diferentes líneas de emisión. Los extremos de la región mapeada fueron elegidos con el propósito de incluir objetos de particular interés, tales como objetos Herbig-Haro en el norte y los choques de proa en el suroeste \citep{Odell:1996}.\\

 Los filtro usados fueron seleccionados de tal manera que se capturaran las líneas de emisión más fuertes, es decir aquellas que representaran un rango razonable de condiciones de ionización. Por tanto se cuenta con un masaico en el que se usó el filtro f658n, es decir en este mosaico domina sólo la presencia de la línea de emisión \nii{} (6584~\A{}). También se cuenta con un mosaico, en el que sólo dominan las líneas de emisión de \ha{} (6563~\A{}), puesto que para capturar las imágenes se usó el filtro f656n. En esta lista se incluye un masaico cuyas imágenes son el resultado del uso del filtro f502n, donde se han registrado las líneas de emisión de \oiii{} (5007~\A{}), que provienen de regiones con la más alta ionización en la nebulosa. Por último existe un mosaico formado por los campos obtenidos a partir del filtro f547m \citep{Burrows:1995}, este es un filtro suficientemente ancho (en comparación a los demás filtros), ubicado en una región del espectro electromagnético, donde no hay líneas fuertes de emisión, por tanto este mosaico son unas imágenes del continuo en esta zona de la Nebulosa de Orión. Los tiempos de exposición para cada filtro son: f656n;  200~s, f658n; 500~s, f502n; 200~s, f547m; 50~s para el programa GO-5085. 



% \bibliography{luis-ref}

% \end{document}
